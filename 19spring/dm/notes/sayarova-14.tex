\documentclass{article}
 
%Russian-specific packages
%--------------------------------------
\usepackage[T2A]{fontenc}
\usepackage[utf8]{inputenc}
\usepackage[russian]{babel}
\usepackage{cancel}
\usepackage{graphicx}
\graphicspath{ {images/} }
\usepackage{amsmath}
%--------------------------------------
 
%Hyphenation rules
%--------------------------------------
\usepackage{hyphenat}
\hyphenation{ма-те-ма-ти-ка вос-ста-нав-ли-вать}
%--------------------------------------
 
 \title{Лекция по дискретной математике}
 \date{6 мая 2019}
 
\begin{document}
 
\maketitle
 
 
 \textbf{Комбинаторика}
 
 Комбинаторика - наука о том, как посчитать количество элементов конечного множества.
 
\underline{Правило сложения}
$|A \cup B|=|A|+ |B|,если A \cap B \cancel = \o$
где $\cup$ - объединение, |A| - количество элементов множества А.

Примеры:
\begin{itemize}
	\item 10 синих шаров = А, 10 красных шаров = В
	
Всего шаров = |A|+ |B| = 20
\item А = двузначные числа с последней цифрой 1
В = двузначные числа с последней цифрой 3

$|A \cup B|$ = двузначные числа, оканчивающиеся 1 или 3

$|A| = |B|=9$. 
$|A \cup B|=9+9=18$
\end{itemize}

\underline{Правило умножения}

$A \times B$ – декартово произведение множеств, или множество пар (a, b), где   
 $a \in A$, $b \in B$;  	$\{(a,b)|a \in A,b \in B\}$

Пример:

A = \{1, 2, 3\}

B = \{x, y\}

$A \times B$ = \{1x,1y,2x,2y,3x,3y\}

$|A \times B|=|A| \ast |B|$

Как видно из предыдущего примера, |A| = 3,  |B| = 2,   $|A \times B|=3 \ast 2 = 6$

Пример:

Сколько всего трехзначных чисел?

|A| = \{1, 2, 3, 4, 5, 6, 7, 8, 9\}                |A| = 9

|B| = \{0, 1, 2, 3, 4, 5, 6, 7, 8, 9\}            |B| = 10

|C| = \{0, 1, 2, 3, 4, 5, 6, 7, 8, 9\}            |C| = 10

$A \times B \times $ $C$- трехзначные числа

$((a, b) , c)   	\Leftrightarrow    (a, b, c)$ – множество троек

$| A \times B \times$ $C$ $|$ $= 9\ast 10 \ast 10 = 900$

$\Rightarrow$	всего 900 трехзначных чисел



\underline{Принцип биекции}

$|A| = |B|  \Leftrightarrow$  существуют взаимно однозначное отображение, которое элементам одного множества сопоставляет элементы другого.

$f:   A \rightarrow B$ 
\begin{enumerate}
	\item $f(A)= B$ или $\forall b\in B \exists a : f(a) = b$
	\item $f(a_1 ) \cancel = f(a_2 )$,если $a_1 \cancel = a_2$  
\end{enumerate}


Пример:
\begin{enumerate}
\item  {a, b, c} {1, 2, 3}

$1 \rightarrow a $

$2 \rightarrow b $

$3 \rightarrow c $

\item 	слова в алфавите a, b, c длины 5 и числа от 0 до 242

 $aaaaa \leftrightarrow 00000_3 = 0$
 
$abacc \leftrightarrow 01022_3 = 27 + 2 \ast 3 + 2 = 35 $

$baccc \leftrightarrow 10222_3 $

$ccacc \leftrightarrow 22022_3  $

$ccccc \leftrightarrow 22222_3 = 242 $

от 0 до 242 $\Rightarrow$  243 числа 

$A = \{a, b, c\}$

$|A \times A \times A \times A \times A|=|A|^5= 3^5=243 $

\item A = множество правильных скобочных последовательностей

B = множество разбиений круга хордами

\end{enumerate}

\underline{Число сочетаний}

Дано множество |A| = n. Сколько в нем подмножеств размера k?

Пример: |A| = 4, k = 2

$A = \{a, b, c, d\}$

$\{ab\}, \{ac\}, \{ad\}, \{bc\}, \{bd\}, \{cd\}$ – 6 шт

Примеры задач

\begin{itemize}
    \item сколько способов выбрать из 100 студентов 10, которые получат 5 автоматом?
\item сколько существует пятизначных чисел, у которых цифры возрастают? (пр. 13789, 23579)

$13789 \rightarrow \{1, 3, 7, 8, 9\}$

$23579 \rightarrow \{2, 3, 5, 7, 9\}$ подмножество $\{1, 2, 3, …, 9\}$ из 5 элементов
\end{itemize}
 
 \underline{Определение}
 $C_n^k$ - количество подмножеств размера k в множестве А.  |A| = n
 
\underline{Вывод формулы:}

Всего k позиций. Существует n способов выбрать 1-ый элемент,  n - 1 – 2-ой элемент, …, n – k + 1 – k-ый элемент.
Всего $n \ast (n-1) \ast \dots \ast (n-k + 1)$ способов выбрать k элементов из n с учетом порядка $$= \frac{n!}{(n-k)!}$$ но порядок не важен

Пусть k = 3

Это все одно и то же:

3711 – 1 способ 

7311 – 1 способ 

1137 – 1 способ 

таких же одинаковых наборов k! штук, это перестановки k элементов
1 способ повторяется $k \ast (k-1) \ast \dots \ast 1 = k!$ раз
т.е. в формуле $\frac{n!}{(n-k)!}$ каждый ответ посчитан $k!$ раз, значит, ответ: $$C_n^k = \frac{n!}{k!(n-k)!}$$

Пример: $C_4^2 = \frac{4!}{2!2!} = \frac{24}{(2 \ast 2)}=6$

\underline{Свойства}
\begin{itemize}
\item	$C_n^k= C_n^(n-k)$
	
Аналитическое доказательство 

$\frac {n!}{k!(n-k)!} = \frac{n!}{(n-k)!k!}$


Комбинаторное доказательство

Множество размера $k \leftrightarrow$ (взаимно отображается) множеством размера $n-k$

n = 9, k = 3

$135 \leftrightarrow 246789$
$B \subset A \leftrightarrow A \setminus B 	\subset A$    

\item	$C_n^0= C_n^n=1$
\item	$C_n^1= C_n^{n-1}=n$
\item	$C_n^k= C_{n-1}^{k-1}  + C_{n-1}^k$

Аналитическое доказательство

$C_{n-1}^{k-1} = \frac{(n-1)!}{(n-1-k+1)!(k-1)!} = \frac{(n-1)!}{(n-k)!(k-1)!} = \frac{(n-1)!}{(n-k)(n-k-1)!(k-1)!}$ 


$C_{n-1}^k = \frac{(n-1)!}{(n-1-k)!k!} = \frac{(n-1)!}{(n-k-1)!k(k-1)!} $

 $C_{n-1}^{k-1} + C_{n-1}^k = \frac{(n-1)!}{(n-k)(n-k-1)!(k-1)!} + \frac{(n-1)!}{(n-k-1)!k(k-1)!} = \frac{(n-1)!k+(n-1)!(n-k)}{(n-k)(n-k-1)!k(k-1)!} = \frac{(n-1)!(k+(n-k))}{(n-k)!k!} = \frac{n!}{(n-k)!k!} = C_n^k$ 

Также это можно заметить при построении треугольника Паскаля

\end{itemize}


\end{document}
