\documentclass{article}
 
%Russian-specific packages
%--------------------------------------
\usepackage[T2A]{fontenc}
\usepackage[utf8]{inputenc}
\usepackage[russian]{babel}
\usepackage{graphicx}
\graphicspath{ {images/} }
\usepackage{amsmath}
\usepackage{amssymb}
\usepackage{cancel}
%--------------------------------------
 
%Hyphenation rules
%--------------------------------------
\usepackage{hyphenat}
\hyphenation{ма-те-ма-ти-ка вос-ста-нав-ли-вать}
%--------------------------------------
 
 \title{Лекция по дискретной математике}
 \date{29 апреля 2019}
 
\begin{document}
 
\maketitle

\textbf{Задача интерполяции}

Найти многочлен $p = F[x], p(x_i) = y_i$, где $x_i, y_i \in F, x_i \cancel = x_j \forall i, j$

К примеру, p(0) = 1, p(1) = 1, p(2) = 3

\underline{Теорема}

Для задачи интерполяции 

$p(x_i) = y_i$, где i = от 1 до n

$\exists$ единственный многочлен p, решение, причем $deg (p) <= n-1$

\underline{Доказательство:}

Единственность.

Пусть p и q оба подходят

$p(x) - q(x)$ имеет корни $x_i$ (n штук)

$p(x) = q(x) = y_i$

$deg (p(x) - q(x)) <= n-1 \Leftrightarrow p(x) - q(x) = 0 $

Существование.

1. Метод Лагранжа

$\sum^n_{i=1} y_i = \frac{\prod^n_{j=1} (x-j)}{\prod^n_{j=1} (x_i - x_j)}$

Пример

p(0) = 1, p(1) = 1, p(2) = 3

$1 \ast \frac{(x-1)(x-2)}{(0-1)(0-2)} + 1 \ast \frac{(x-0)(x-2)}{(1-0)(1-2)} + 3 \ast \frac{(x-0)(x-1)}{(2-0)(2-1)}$


2. Метод Ньютона

Начинать с многочлена $p_1(x_1) = y_1$. Он подходит под одно уравнение - $p_1(x_1) = y_1$.

$p_2(x_2)$ должен подходить под два уравнения. $p_2(x_2) = y_2, p_2(x) = p_1(x) + (x-x_1) \alpha$

$p_{k+1}(x) = p_k(x) + (x-x_1) ... (x-x_k) \alpha$


\textbf{НОД многочленов}

\underline{Определение}

НОД p(x) и q(x) - многочлен d(x):
1) $p(x) \mathop{\raisebox{-2pt}{\vdots}} d(x)$

$q(x) \mathop{\raisebox{-2pt}{\vdots}} d(x)$

2) d(x) - наибольшая степень

Также, например, $d(x) \mathop{\raisebox{-2pt}{\vdots}} \overline{\rm d(x)}$, если $\overline{\rm d(x)}$ - общий делитель p(x) и q(x).

Если $d_1(x)$ и $d_2(x)$ - НОДы p(x) и q(x), то $d_1(x)\mathop{\raisebox{-2pt}{\vdots}} d_2(x)$ и $d_2(x)\mathop{\raisebox{-2pt}{\vdots}} d_1(x)$

$\Rightarrow d_1(x) = d_2(x) \ast u(x)$, где deg(u(x)) = 0 (число)

Все НОД отличаются домножением на константу.

Можно рассматривать только d(x) со старшим коэффициентом, равном единице

Пример

$5x^2-6x+1, 3x^2+2x-5$

$\frac{5x^2-6x+1}{3x^2+2x-5} = \frac{5}{3} - \frac{28}{3}x + \frac{28}{3} $

$5x^2-6x+1 (1, 0) 3x^2+2x-5 (0, 1)$

$- \frac{28}{3}x + \frac{28}{3} (1, - \frac{5}{3}) 3x^2+2x-5 (0, 1)$

НОД (p(x), q(x)) = НОД ($\alpha$ p(x), $\beta q(x))$

Ответ: $-x+1 = \frac{3}{28}5x^2-6x+1 - \frac{5}{28}(3x^2+2x-5)$

Аналогично числам, есть простые многочлены, или неприводные, их нельзя представить в виде произведения меньших степеней

Неприводимость 

Приводимость зависит от поля 

$x^2+1$ над $\mathbb {R}$ неприводим

$x^2+1$ над С приводим: $x^2+1 = (x+i)(x-i)$

$x^2+1$ над $\mathbb {Z}_2$ приводим: $x^2+1 = (x+1)(x+1)$

\underline{Утверждение}

Над С неприводимы многочлены только первой степени $\alpha x + \beta, \alpha \cancel = 0$

\underline{Утверждение}

Над $\mathbb {R}$ неприводимы только $\alpha x + \beta, \alpha x^2 + \beta x + \gamma$, где $\beta^2 - 4 \alpha \gamma < 0$

\underline{Утверждение}

Приводимость над $\mathbb {Q} \Leftrightarrow$ над $\mathbb {Z}$

Доказательство:

$p(x) = q(x) \ast r(x)$
p(x) имеет целые коэффициенты
$q(x), r(x)$ имеют вещественные коэффициенты

\underline{Утверждение}

f(x) с целыми коэф. проводим в $\mathbb {Q} \Rightarrow $ f(x) приводим в $\mathbb {Z}_p$, где старший коэффициент $f  \cancel{ \mathop{\raisebox{-2pt}{\vdots}}} p$ 

Доказательство

f(x) = g(x)h(x). Рассмотрим по модулю p

$f(x)  \equiv g(x)h(x) \pmod{p}$

f(x) той же степени, что и над Q[x]

Критерий Эйзенштейна

приводимость над $\mathbb {Z} (Q)$

$a_n x^n + ... + a_1 x + a_0, p \in \mathbb {P}$

Если $a_n \cancel {\mathop{\raisebox{-2pt}{\vdots}}} p, a_i \mathop{\raisebox{-2pt}{\vdots}} p, a_0 \cancel {\mathop{\raisebox{-2pt}{\vdots}}} p^2 $

$\Rightarrow $ многочлен неприводим

Пример:

$x^3 + 2x^2 + 4x - 2$ неприводим (при p = 2)

$x^n \pm 2 $ неприводим

Доказательство:

$a_n x^n + ... + a_0 = (b_m x^m + ... + b_0) \ast (c_k x^k + ... + c_0)$

$b_0 c_0  = a_0 \mathop{\raisebox{-2pt}{\vdots}} p \Rightarrow b_0$ или $c_0 \mathop{\raisebox{-2pt}{\vdots}} p (a_0 \cancel {\mathop{\raisebox{-2pt}{\vdots}}} p^2) $

$b_0 c_1 + b_1 c_0 = a_1 \mathop{\raisebox{-2pt}{\vdots}} p$

$b_0 c_1 \mathop{\raisebox{-2pt}{\vdots}} p \Rightarrow b_1 c_0 \mathop{\raisebox{-2pt}{\vdots}} p \Rightarrow b_1 \mathop{\raisebox{-2pt}{\vdots}} p$

$\Rightarrow b_i \mathop{\raisebox{-2pt}{\vdots}} p \forall i \Rightarrow a_n \mathop{\raisebox{-2pt}{\vdots}} p$

Над $\mathbb {Z}_2$ неприводимы x+1, x



\end{document}
