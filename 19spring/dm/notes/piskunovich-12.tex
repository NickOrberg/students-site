\documentclass{article}
 
%Russian-specific packages
%--------------------------------------
\usepackage[T2A]{fontenc}
\usepackage[utf8]{inputenc}
\usepackage[russian]{babel}
\usepackage{cancel}
\usepackage{upgreek}
\usepackage{ dsfont }
\usepackage{ amssymb }
\usepackage{graphicx}
\graphicspath{ {images/} }
\usepackage{amsmath}
%--------------------------------------
 
%Hyphenation rules
%--------------------------------------
\usepackage{hyphenat}
\hyphenation{ма-те-ма-ти-ка вос-ста-нав-ли-вать}
%--------------------------------------
 
 \title{Лекция по дискретной математике}
 \date{22 апреля 2019}
 
\begin{document}
 
\maketitle
 
 
 \textbf{Задача интерполяции}
 
Поле $\mathds{R},\mathds{C},\mathds{Q},\mathds{Z_{p}}$

\underline{Задача интерполяции}

Разложить $ X^4+x^3-x-1$ в $\mathds{R}$

Найти мн-н p, такой что $p(x_{i})=y_{i}$, где $x_{i},y_{i} \in F$ $x_{i} \neq x_{j}$ при $i\neq j$

Например: $p(0)=1,p(1)=1,p(2)=3$

\textbf{Задача интерполяции}

Для задачи интерполяции $p(x_{i})=y_{i}$, где $1\leq i \leq n \exists !$ мн-н p, решение $deg p\leq n-1$

Д-во: 
1)!

p и q подходят $\underbrace{p(x)-q(x)}$ имеет корни $x_{i}$ $p(x_{i}=q(x{_i})=y_{i} \Rightarrow$

$p(x)-q(x)=0$ (нулевой многочлен)

Замечание: если степень не ограничить, то $p(x)-q(x)=(x-x_{1})(x-x_{1})...(\mathcal{Y}-x_{n})\mathcal{Y}(\lambda)$

т.е. если p(x) подходит, то $p(x)=p_{0}(x)+\sqcap^n_{n=1}(x-x_{1})r(x)$ при этом $deg\leq n-1$

2) $\exists$

\textbf{1.Метод Лагранжа явная ф-ма:}
 
$\Sigma^i_{i-1}$ $$\underbrace{ y_{i}\frac{\sqcap^n_{j=1, j\neq i}(x-x_{j})}{\sqcap^n_{j=1, j\neq i}(x-x_{j})}}_{p_{i}(x)}$$

Пример: $p(0)=1; p(1)=1; p(2)=3$

\begin{enumerate}

\item $\underbrace{\frac{(x-1)(x-2)}{(0-1)(0-2)}}_{p_1(x)}+1\underbrace{\frac{(x-0)(x-2)}{(1-0)(1-2)}}_{p_2(x)}+3\underbrace{\frac{(x-0)(x-1)}{(2-0)(2-1)}}_{p_3(x)}$ 

\end{enumerate}

Ответ: $x^2-x+1$

почему подходит условие? 

$p_i(x),p_i(x_i)=1,p_i(x_j)=0$

Ответ: $y_{1}P_1(x)+y_{2}P_2(x)+...+y_{n}P_n(x)$ подставим $x-x_i\Rightarrow$

$y_{1}0+y_{2}0+...+y_{n}0\pm9$

\textbf{2.Метод Ньютона:}

Начинаем с мн-на $p_1(x)-y_1$ он подходит под $p_1(x_1)=y_1$-одно ур-ие

$p_2(x)$-должен подойти под 2 ур-ия $(q+(x_1)=y_1,p_2(x_1)=y_2)$

$p_2(x)-p_1(x)+(x-x_1)\alpha$ подбираем $\alpha$

$p_{k+1}(x)=p_{k}+(x-x_1)...(x-x_k)\alpha$

\underline{Определение}

Нод-p(x) и q(x)-это мн-н d(x)

1) $p(x)\vdots d(x)$ |

   $q(x)\vdots d(x)$ | $p(x)=d(x+u(x))$

2) d(x)-наиб степень

Все старые Th. сокращаются

Например: $d(x)\vdots d(x)$, если d(x)-общий делитель p(x) и q(x)

Если $d_1(x)$ и $d_2(x)$ - НОД p(x) и q(x), то $d_1(x)\vdots d_2(x)$ и 
$d_2(x)\vdots d_1(x) \Rightarrow d_1(x)=d_2(x)*u(x)$

Все НОД отличаются умножением на const, т.е $d_1(x)\backsim d_2(x)$

Можно расшифровать только d(x) со старшим коэф.=1 Нод$(...)=x+1$

Алгоритм Евклида для мн-в не отличается от обычного.

Пример: 

$5x^2-6x+1$ $|$ $3x^2+2x-5$ в $\mathds{R}$

Числитель: $5x^2-6x+1$

Делитель: $3x^2+2x-5$

Остаток: $\frac{28}{9}x+\frac{13}{9}$

Ответ: $\frac{5}{3}$

$5x^2-6x+1$(1,0) $3x^2+2x-5$(0,1)

$-\frac{28}{3}x+\frac{28}{3}$(1,-(5/3) $3x^2+2x-5$(0,1)

НОД (p(x),q(x))-НОД $(\alpha p(x),\beta q(x))$ $\alpha,\beta$-число

$-x+1(3/28;-(5/28))$

$-x+1(2/28;-(5/28))$

Ответ: $-x+1=\frac{3}{28}$ $(5x^2-6x+1)$ $-\frac{5}{28}$ $(3x^2+2x-5)$

Аналогично числам, есть простые мн-ны-неприводимые.

Неприводимые мн-н нельзя представить в виде произведения двух мн-ов строго меньших степеней. 

Неприводимость-трудно проверить в общем случае.

$x^2+1$ над $\mathds{R}$ $x^2+1=(x+\alpha)(x+\beta)$

$x^2+1$ над $\mathds{C}$ $x^2+1=(x+i)(x-i)$

$x^2+1$ над $\mathds{Z_p}$ $x^2+1=(x+1)(x-1)$

\underline{Утверждение 1.}

НОД $\mathds{C}$ неприводимы только мн-ны первой степени $\alpha x+\beta$ $\alpha\neq 0$

\underline{Утверждение 2.}

НОД $\mathds{R}$ неприводимы только мн-ны $\alpha x+\beta$ и $\alpha x^2+\beta x+\gamma$, где $\beta^2-4\alpha\gamma=0$

\underline{Утверждение 3.}

Приводимость над $\mathds{Q}\Leftrightarrow \mathds{Z}$

Д-во:

$p(x)=q(x)*\gamma(x)$

$p(x)=\frac{1}{MN}(M_p(x))(N_q(x))$

рассмотреть k простой делитель из MN 

\underline{Утверждение} $f(x)$ приводим в $ \mathds{Q}\Rightarrow f(x)$ приводим в $\mathds{Z_p}$

Д-во: $f(x)=g(x)h(x)$ целый коэф. рассмотрим по mod p 

$f(x)\stackrel{\mathrm{p}}{\equiv}g(x)h(x)$

\textbf{Критерий Эйзенштейна}

Приводимость над $\mathds{z(\mathds{c})}$

$a_{n}x^n+a_{1}x+a_0$

если $a_n\vdots d p$, $a_1\vdots p$, $a_0\vdots p^2$ мн-н неприводим

Пример: $x^3+2x^2+4x-2$ непр.(при $p=2$)

Д-во: $a_{n}x^x+\dots+a_0=b_{m}x^n+\dots+b_0=c_{k}x^x+\dots c_0$

$b_{0}c_{0}+a_{0}ip\Rightarrow b_0$ или $c_0\vdots p$

\end{document}