\documentclass{article}
\usepackage[utf8]{inputenc}
\usepackage[T2A]{fontenc}
\usepackage[russian]{babel}
\usepackage{amssymb}
\usepackage{graphicx}
\usepackage{amsmath}
\DeclareRobustCommand{\divby}{%
     \mathrel{\text{\vbox{\baselineskip.65ex\lineskiplimit0pt\hbox{.}\hbox{.}\hbox{.}}}}}% 

\title{Лекция 8}
\date{}

\begin{document}

\maketitle

\section{Исчисление предикатов}
Исчисление высказываний есть функция логическая. Переменные в функциях пропозициональные - 0 или 1, легко перебирать все возможные значения переменных. \\*
Исчисления предикатов : переменные - предметные , принимают значения = элементы ненулевого множества, т.е. можно формулировать утверждения про элементы какого-то множества (числа, студенты, слова...) \\*
Итак, чтобы понять как формулировать утверждения, заводим множество M $\neq 0$ \\*
\underline{Определение}: Предикат - это функция $P:M^K \longrightarrow B$ n$\geq 0 ; k \in Z$ \\*
B={0 , 1} (ложь и истина) \\*
Примеры : \\*
M = Z \\*
$P_1(x) = x \geq 0 $ (что х положительно) \\*
$P_2(x) = x^2 + 1 \geq 7$ \\*
$P_3(x) = x $ содержит цифру 1 в десятичной записи \\*
$P_3(238) = 0 , P_3(-571) = 1$ \\*
$P_4(x,y) = x > y$ \\*
$P_5(x,y) = x^2 + y^2 = 25 (P_5(7,8) = 0 , P_5(3,4) = 1 , P_5(0,5) = 1 )$ \\*
$P_6 (k = 0)$ , нет переменных) = 0 \\*
$P_7 = 1$ \\*
$P_8(x,y) = x$ посещал лекции чаще чем y в этом семестре \\*
М = студенты этого потока \\*
\underline{Определение}: Функции  - f: $M^k \longrightarrow M (k \geq 0 , k \in Z)$ \\*
Функции превращают один или несколько элементов множества в элемент множества \\*
Пример : 
M = Z \\*
$f_1(x,y) = x+y$ \\*
$f_2(x,y) = x^2 + y - 1$ \\*
\begin{equation*}
f_3(x,y) = 
\begin{cases}
$y, если х - четное$ \\
$42 ,  если х -нечетное$ \\
\end{cases}
\end{equation*}\\*
$f_3(2,5) = 5 ; f_3(7,8) = 42$ \\*
$f_4(x) = x^2$ \\*
$f_5(x) = $ x без цифр 1 в десятичной записи \\*
$f_5(42) = 42 ; f_5(57121) = 572 ; f_5(111) = 0$ \\*
$f_6 = 7$ = константа (k = 0) \\*
$f_7(x,y) = x $ если х = у \\*
М = {Ниф- ниф, Наф-Наф, Нуф-Нуф} = третий студент (кроме х,у) \\*
Замечание : \\*
Предикаты - заглавные буквы P,Q,R,S,A,B,C \\*
Функции - строчные буквы f,g,h \\*
константы - начало алфавита a,b,c... \\*
2) Некоторые функции и предикаты можно записать привычно в инфиксной форме \\*
x>y вместо P(x,y) > (x,y) \\*
x+y вместо f(x,y) + (x,y) \\*
$x^y$ вместо g(x,y) ,  где g возведение в степень \\*
\underline{Определение}: Формула исчисления предикатов. \\*
Содержит : предикатные символы, функциональные символы, предметные переменные, кванторы \\*
Подопределение : 
Терм: переменная функции , символ ($\underbrace { \mbox{, , }}_{\mbox{термы}}$)\\*
Пример : пусть х,у,z - переменные, пусть $\underbrace{\underbrace{f}_{1},\underbrace{g}_{2},\underbrace{a}_{0}}_{\mbox{кол-во аргументов}}$ -  функции \\*
Остальные символы \\*
x  ; f(f(x)) \\*
f(x) ; f(g(x,y)) \\*
f(g(x,g(a,f(y)))) \\*
Формула исчисления предикатов - это \\*
$\fbox{1}$ - предикатный символ от нескольких термов (терм, терм, терм) \\*
$\fbox{2}$ - $\forall x$ формула исчисления предикатов со свободной перменнной х \\*
$\fbox{3}$ - $\exists x$ формула исчисления предикатов со свободной переменной х \\*
(1) - все переменные внутри термов свободны \\*
(2,3) - здесь переменная х перестает быть свободной, становится связанной \\*
- функция исчисления предикатов ($\widetilde{\mbox{ФИП}}$) , ФИП $\longrightarrow $ ФИП \\*
Замечание : ФИП - это выражение предиката через другие предикаты и функции. При этом смысл предикатов не важен. \\*
Но если смысл будет задан, то получится конкретный предикат \\*
Примеры : 
1) $\underbrace{P}_{\mbox{предикатный символ}} (\underbrace{x}_{\mbox{терм(переменная)}},\underbrace{y}_{\mbox{терм(переменная)}})$   (P(x,y)) \\*
Интерпретация : \\*
Задать М и смысл P \\*
1)M = Z P(x,y) : x > y \\*
2)M = студенты P(x,y) : х чаще ходит \\*
Обе переменные х,у - свободны \\*
Это значит, что им можно назначить какое-то значение, и тогда результат вычисления формулы истина или ложь \\*
x = 5 , y = 7 P(x,y) = 0  \\*
x = 4 , y = 2 P(x,y) = 1 \\*
2. $\underbrace{P}_{\mbox{предикатный символ}} (\underbrace{x}_{\mbox{терм(переменная)}},\underbrace{a}_{\mbox{терм(константа)}})$ - P(x,a) \\*
Интерпретация : P - ? , M - ? , a - ? \\*
1)M = Z , P(x,y) : x>y , a = 7 \\*
Одна свободная переменная х \\*
x = 8 : 1 \\*
x = 6 : 0 \\*
3. P(x) $\vee$ Q(x) \\*
Интерпретация P = ?, Q = ? , M = ? \\*
M = Z ; P(x) = x - четное ; Q(x) = х - нечётное \\*
свободные переменные : x \\*
x = 1 : 0 $\vee$ 1 = 1 \\*
x = 1 : 1 $\vee$ 0 = 1 \\*
x = неважно : ... = 1 \\*
4.$\forall x$ P(x,y) \\*
Интерпретация : \\*
M = ? P = ? \\*
свободны : y (x - связан) \\*
вычисляем \\*
1) M = Z , P(x,y) = x $\geq$ y \\*
y = 0 $\forall$ x P(x,0) $\leftrightarrow \forall$ x $x \geq 0$ \\*
Чтобы вычислить $\forall x P(x) $ надо проверить , что P(x) всегда 1 при всех $x \in M$. Результат 0 \\*
2)M = N ; P(x,y) : $x \geq y$ \\*
$\forall x $ P(x,y) \\*
при y = 1 $\rightarrow$ 1 \\*
при y = 2 $\rightarrow$ 0 \\*
Замечание : Чтобы вычислить значение формулы исчисления предиката надо \\*
-Интерпретация, то есть M = ? P,Q = ? f,g = ? \\*
-Задать смысл предикатов и функциональных символов \\*
-Задать значения свободных переменных \\*
Из этого мы получаем не 0, 1, а предикат, т.е. $\forall$ x P(x,y) = $\underbrace{Q(y)}_{\mbox{предикат от y}}$ \\*
5. $\exists x \forall y P(x,y)$ \\*
Свободные переменные : нет \\*
Интерпретация : \\*
1)M = Z ; P(x,y) : x $\leq$ y \\*
$\forall y $ P(x,y) = Q(x) \\*
или Q(x) $\forall y x \leq y$ \\*
Q(0) $\forall y 0 \leq y$ = ложь (0) \\*
Q(-1) $\forall y -1 \leq y$ = ложь (0) \\*
Q(...) $\forall y ... \leq y$ = ложь (0) \\*
т.е. Q(x) = 0 независимо от x \\*
$\exists x \underbrace{\forall y P(x,y)}_{Q(x)} = \underbrace{\forall x}_{\mbox{можно подобрать x} \in M : \mbox{внутри 1}} 0$ \\*
2)Другая интерпретация \\*
M = N ; P(x,y) : x $\leq $ y\\*
Q(1) : $\forall y 1 \leq y = T$ \\*
Q(2) : $\forall y 2 \leq y = F$ \\*
Q(...) = F \\*
\begin{equation*}
\exists x \underbrace{\forall y P(x,y)}_{Q(x)}  =
\begin{cases}
x = 1 : T \\
x = $иначе$ : F\\
\end{cases}
\end{equation*}\\*
= T (при х = 1) \\*
Еще примеры : начнем с интерпретации  \\*
M = N \\*
P(x,y) : x = y\\*
f(x,y) : x+y \\*
и еще несколько стандартных \\*
x > y $\exists k$ x = y + k \\*
$x \divby y = \exists k$ x =y*k \\*


 
\end{document}
