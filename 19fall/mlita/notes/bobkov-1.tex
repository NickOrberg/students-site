\documentclass{article}
 
%Russian-specific packages
%--------------------------------------
\usepackage[T2A]{fontenc}
\usepackage[utf8]{inputenc}
\usepackage[russian]{babel}
\usepackage{cancel}
\usepackage{graphicx}
\graphicspath{ {images/} }
\usepackage{amsmath}
%--------------------------------------
 
%Hyphenation rules
%--------------------------------------
\usepackage{hyphenat}
\hyphenation{ма-те-ма-ти-ка вос-ста-нав-ли-вать}
%--------------------------------------
 
 \title{Лекция по математической логике и теории алгоритмов}
 \date{4 сентября 2019}
 
\begin{document}
 
\maketitle
$$\\$$
 \textbf{Исчисление высказываний}
 
 Логическая формула: выражение со значениями (0,1)
 
                    \hspace{35mm} переменными (x,y, ...)

    \hspace{35mm} и операциями (•,  $\vee$,  $\leftrightarrow$, ...)
\newline
\underline{Арифметические выражения.}

Пример: ((z+x)•y-10)/z
\newline
\underline{Логические выражения.}

Пример: (1+$\neg$x)$\rightarrow$(x*y $\leftrightarrow$ $\neg$y $\neg$z)

где 1 - значение; x,y - переменные; $\leftrightarrow$ - операция.

Значения: 0 - ложь(F), 1 - истина(T).
\newline
\underline{Операции. Унарная операция.}

$\neg$ - отрицание.

$\neg$1 = 0.

\begin{tabular}{|c|c|c|}
\hline
x & $\neg$x \\\hline
0 & 1 \\\hline
1 & 0\\\hline
\end{tabular}
$$\\$$
\underline{Бинарные операции(2*2*2*2=16 - всего операций.)}

\begin{tabular}{|cc||*{7}{c|}}
\hline
x & y & x $\vee$ y & x $\wedge$ y & x $\oplus$ y &
x $\rightarrow$ y & x $\equiv$ y & x | y & x $\downarrow$ y\\
\hline
0 & 0 & 0 & 0 & 0 & 1 & 1 & 1 & 1 \\
0 & 1 & 1 & 0 & 1 & 1 & 0 & 1 & 0 \\
1 & 0 & 1 & 0 & 1 & 0 & 0 & 1 & 0 \\
1 & 1 & 1 & 1 & 0 & 1 & 1 & 0 & 0 \\ \hline
\end{tabular}
\newline
Пример импликации:

1) 2*2=5 $\Longrightarrow$ сегодня суббота

Верно

2) $\lnot$ (1 $\Longrightarrow$ 0) *1 = $\lnot$ 0 *1= 1*1=1

Свойства операций:

$\wedge$,$\vee$,$\leftrightarrow$,+ - коньюктивны

x*y=y*x - умножение

x $\vee$ y = y $\vee$ x

x $\leftrightarrow$ y = y $\leftrightarrow$ x - равенство

x + y = y + x - Сумма в Z2

Универсальный способ проверки равенства двух логических выражений, это сравнить значения выражений при всех возможных значениях переменных.

\begin{tabular}{|c|c|c|c|}
\hline
x & y & x $\vee$ y & y $\vee$ x \\\hline
0 & 0 & 0 $\vee$ 0 = 0 & 0 $\vee$ 0= 0 \\\hline
0 & 1 & 0 $\vee$ 1 = 1 & 1 $\vee$ 0 = 1\\\hline
1 & 0 & 1 & 1\\\hline
1 & 1 & 1 & 1\\\hline
\end{tabular}
$$\\$$
Третий и четвертый столбцы совпадают.
$$\\$$
x $\Longrightarrow$ y - не коммутативно
$$\\$$
\begin{tabular}{|c|c|c|c|}
\hline
x & y & x $\Longrightarrow$ y & y $\Longrightarrow$ x \\\hline
0 & 0 & 0 $\Longrightarrow$ 0 = 1 & 0 $\Longrightarrow$ 0= 1 \\\hline
0 & 1 & 0 $\Longrightarrow$ 1 = 1 & 1 $\Longrightarrow$ 0 = 0\\\hline
\end{tabular}
$$\\$$
Третий и четвертый столбцы не совпадают.
$$\\$$
Ассоциативность:

(x $\vee$ y) $\vee$ z = x $\vee$ (y $\vee$ z)

(x * y)*z = x*(y*z)
$$\\$$
\begin{tabular}{|c|c|c|c|c|c|c|}
\hline
x & y & z & x $\vee$ y & (x $\vee$ y) $\vee$ z & y $\vee$ z & x $\vee$ (y $\vee$ z) \\\hline
0 & 0 & 0 & 0 & 0 & 0 & 0 \\\hline
0 & 0 & 1 & 0 & 1 & 1 & 1 \\\hline
0 & 1 & 0 & 1 & 1 & 1 & 1 \\\hline
0 & 1 & 1 & 1 & 1 & 1 & 1 \\\hline
1 & 0 & 0 & 1 & 1 & 0 & 1 \\\hline
1 & 0 & 1 & 1 & 1 & 1 & 1 \\\hline
1 & 1 & 0 & 1 & 1 & 1 & 1 \\\hline
1 & 1 & 1 & 1 & 1 & 1 & 1 \\\hline
\end{tabular}
$$\\$$
Пятый и седьмой столбцы совпадают.

(x $\leftrightarrow$ y) $\leftrightarrow$ z = x $\leftrightarrow$ (y $\leftrightarrow$ z)

(x $\Longrightarrow$ y) $\Longrightarrow$ z $\neq$ x $\Longrightarrow$ (y $\Longrightarrow$ z), т.к.
$$\\$$
\begin{tabular}{|c|c|c|c|c|}
\hline
x & y & z & (x $\Longrightarrow$ y) $\Longrightarrow$ z & x $\Longrightarrow$ (y $\Longrightarrow$ z)\\\hline
0 & 0 & 0 & (0 $\Longrightarrow$ 0) $\Longrightarrow$ 0=0 & 0 $\Longrightarrow$ (0 $\Longrightarrow$ 0)=1 \\\hline
\end{tabular}
\newline
\underline{Запись логических выражений.}

У ассоциативных операций () можно не иметь.

x+y+z - нормально.

Пример: x$\rightarrow$y$\rightarrow$z

Если скобок нет, имеется в виду x$\rightarrow$(y$\rightarrow$z).

Приоритет операций x$\vee$y*z $\rightarrow$ $\neg$(x*y),

1)$\neg$ 

2) • $\wedge$

3) $\vee$+ 

4) $\leftrightarrow$ и $\rightarrow$
\newline
\underline{Правило Дэ Моргана}

$\neg$(x$\vee$y) = $\neg$x $\wedge$ $\neg$y

$\neg$(x$\wedge$y) = $\neg$x $\vee$ $\neg$y

Проверка:

\begin{tabular}{|c|c|c|c|}
\hline
x & y & $\neg$(x$\wedge$y) & $\neg$x $\vee$ $\neg$y \\\hline
0 & 0 & 1 & 1 \\\hline
0 & 1 & 1 & 1\\\hline
1 & 0 & 1 & 1\\\hline
1 & 1 & 0 & 0\\\hline
\end{tabular}
\newline
\underline{Дистрибутивность.}

x•(y$\vee$z) = x•y $\vee$ x•z

x•(y+z) = x•y + x•z

x $\vee$ (y•z) = (x$\vee$y)•(x$\vee$z) - двойственная

x + y•z $\neq$ (x+y)(x+z) неверно при x=1, y=1, z=0.
\newline
\underline{Ещё набор свойств:} 

$\neg\neg$x = x

x $\rightarrow$ y = $\neg$x $\vee$ y - см. таблицу истинности.

x $\leftrightarrow$ y = (x$\rightarrow$y)(y$\rightarrow$x) - см. Мат. Анализ.
\newline
\underline{Дизъюктивно-нормальная форма.}

Нормальная форма - один из вариантов записи логических выражений.

x•y$\vee$z = (x$\vee$z)(y$\vee$z) = x•y$\vee$z$\vee$0 = x•y+z+x•y•z

Здесь x•y$\vee$z - дизъюктивно-нормальная форма(ДНФ)

Выражение имеет ДНФ, если оно является дизъюнкцией нескольких конъюктов. 

Конъюнкт - это конъюнкция литералов.

Литерал - переменная или отрицание переменной.

Пример: x•y$\vee$z

где z - литерал; x и y - конъюнкты.

Ещё ДНФ: x•$\neg$y•z $\vee$ x•$\neg$y•$\neg$z $\vee$ $\neg$y•z - 3 конъюнкта.

x•$\neg$y•z - 1 конъюнкт.

$\neg$x $\vee$ $\neg$y $\vee$ z - 3 конъюнкта по 1 литералу.

$\neg$x - ДНФ, 1 конъюнкт из 1 литерала.

x•$\neg$y•z $\vee$ $\neg$y•z $\vee$x $\vee$ $\neg$y•$\neg$z

Не ДНФ:

x$\vee$1,

x$\vee$y$\vee$z$\vee$x$\rightarrow$y,

x$\vee$y•z,

x+y.




\end{document}
