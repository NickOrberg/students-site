\documentclass{article}
\usepackage[utf8]{inputenc}
\usepackage{amsmath}
\usepackage{amssymb}
\usepackage{multirow}
\usepackage{ dsfont }
\usepackage[russian]{babel}

\usepackage[left=2cm,right=2cm,
    top=2cm,bottom=2cm,bindingoffset=0cm]{geometry}

\title{Лекция 8}
\date{30 октября 2019}

\begin{document}

\maketitle

\section{Исчисление предикатов}

Исчисление высказываний:

есть логическая функция,

переменные в функциях пропорциональные: 0 или 1.

\noindent Легко перебирать все возможные значения переменных.

~\\
Исчисление предикатов: переменные - \textbf{предметные}.

принимают значения = элементы непустого множества, т.е. можно формулировать утверждения про элементы какого-то множества (числа, студенты, слова, ...).

Итак, чтобы начать формулировать утверждения, заводим множество $M\neq \varnothing.$

~\\
\textbf{Опр. }Предикат - это функция $P: M^k \rightarrow \mathds{B}$

$\mathds{B} = \{\mathds{O}, \mathds{1}\}$, где $\mathds{O}$ - ложь, $\mathds{1}$ - истина.

~\\
\textbf{Примеры. }

$M = \mathds{Z}$

$P_1(x) = x\geqslant 0$ - предикат: число положительное.

$P_2(x) =  x^2+1\geqslant 7$

$P_3(x) = x$ содержит 1 в десятичной записи. 

Например, $P_3(238) = \mathds{O}, P_3(-571) = \mathds{1}$

$P_4(x,y) = x>y$

$P_5(x,y) = x^2+y^2 = 25$

Например, $P_5(7,8)=\mathds{O}, P_5(3,4)=\mathds{1}, P_5(0,5)=\mathds{1}.$

$P_6 = \mathds{O}$ - k = 0 (нет переменных).

$P_7 = \mathds{1}$

$P_8(x,y) = x$ посещал лекции чаще $y$ в этом семестре.
М - студенты этого потока.

~\\
\textbf{Опр.} Функции. $f: M^k \rightarrow M, k\geqslant 0, k \in \mathds{Z}.$

Функции превращают один или несколько элементов множества в элемент множества.

~\\
\textbf{Примеры.} $M = \mathds{Z}$

$f_1(x,y) = x+y$

$f_2(x,y) = x^2+y-1$

$f_3(x,y) = 
\left\{ \begin{array}{ll}
         y & \mbox{если $x$ - четное};\\
        42 & \mbox{если $x$ - нечетное}.\end{array} \right. $
        
Например, $f_3(2,5) = 5, f_3(7,8)=42$.

$f_4(x)=x^2$

$f_5(x) = x$ без цифр 1 в 10-й записи. 
$f_5(412) = 42, f_5(57121)=572, f_5(111) = 0.$

$f_6 = 7, k=0$ - константа.

На множестве M=$\{$Ниф-Ниф, Наф-Наф, Нуф-Нуф$\}$.

$f_7(x,y) =$ третий студент (кроме х и у).

Обязательно нужно доопределить $f_7(x,y) = x$, если $x=y$.


~\\
\textbf{Замечания.}

1)предикаты - заглавные буквы P, Q, R, A, B, ...

функции - строчные буквы f, g, h, ...

константы - начало алфавита a, b, c, ...

2) некоторые функции и предикаты можно записать привычно в инфиксной форме: $x>y$ вместо $P(x,y):\hspace{10pt} >(x,y)$.

$x+y$ вместо $+(x,y)$.

$x^y$ вместо $g(x,y)$, где $g$ - возведение в степень.


~\\
\textbf{Опр.} Формула исчисления предикатов.

Содержит:

\hspace{10 pt}предикатные символы

\hspace{10 pt}функциональные символы

\hspace{10 pt}предметные переменные

\hspace{10 pt}кванторы

\noindent\rule{\textwidth}{0.01pt}
\textbf{Подопределение.}

Терм:

\hspace{10 pt}переменная

\hspace{10 pt}функциональный символ (..., ..., ...) - внутри скобок находятся термы.

\noindent\textbf{Пример.} ] $x, y, z $ - переменные.

] $f, g, a $ - функциональные символы. Количество аргументов: для $f = 1, g=2, a=0$.

$x \hspace{30pt} f(f(x)) \hspace{30pt} f(g(x,g(a,f(y))))$

$f(x) \hspace{15pt} f(g(x,y))$

\noindent\rule{\textwidth}{0.01pt}

~\\
Формула исчисления предикатов - 

а) это предикатный символ(терм, терм, ...) от нескольких термов.

б) $\forall x$ формула исчисления предикатов со свободной переменной $x$.

в) $\exists x$ формула исчисления предикатов со свободной переменной $x$.

Все переменные внутри термов свободны.

В последних двух пунктах переменная х перестает быть свободной - становится связной.

$\neg$ФИП, ФИП $\Rightarrow$ ФИП

~\\
\textbf{Замечание}. ФИП - это выражение предиката через другие предикаты и функции. При этом смысл функций и предикатов не важен. Но если смысл будет задан, то получится конкретный предикат.

~\\
\textbf{Примеры. }

\noindent 1.P(x,y), где х и у - термы (переменные), P - предикатный символ.

Интерпретация: задать М и смысл Р.

1)$M=\mathds{Z} \hspace{50pt} P(x,y): x>y$

2)$M$ - студентов \hspace{10pt} $P(x,y):$ х чаще ходит

Обе переменные х, у свободны.

Это значит, что им можно назначить какое-то значение, и тогда результат вычисления формулы - истина или ложь.

$x=5, y=7. \hspace{10pt} P(x,y)=\mathds{O}$

$x=4, y=2. \hspace{10pt} P(x,y)=\mathds{1}$

\noindent 2. P(x,a), где х - переменная (терм), а - константа (терм), P - предикатный символ.

Интерпретация: Р-?, а-?, М-?

1) $M=\mathds{Z}, a=7$

$x=8: \mathds{1}$

$x=6: \mathds{O}$

одна свободная переменная х.

\noindent 3. $P(x) \vee Q(x)$

Интерпретация.  Р-?, Q-?, М-?

1) $M=\mathds{Z}$

$P(x)= x$ - четное

$Q(x) = x$ - нечетное

Свободные переменные: х.

$x=1 \hspace{10pt} \mathds{O}\vee \mathds{1}=\mathds{1}$

$x=10 \hspace{10pt} \mathds{1}\vee \mathds{O}=\mathds{1}$

х - неважно. Результат = $\mathds{1}$.

\noindent 4. $\forall x \hspace{5pt} P(x,y)$

Интерпретация: М-? Р-?

Свободны: у (х - связан).

1)$M=\mathds{Z}, P(x,y): x\geqslant y$

$y=0 \hspace{5pt} \forall x \hspace{5pt} P(x,0) \Leftrightarrow \forall x \hspace{5pt} x\geqslant 0$. Результат = $\mathds{O}$.

Чтобы вычислить $\forall x \hspace{10pt} P(x) $ надо проверить, что $P(x)$ всегда $\mathds{1}$ при всех $x \in M$.

2) $M=\mathds{N} \hspace{5pt} P(x,y): x\geqslant y$

$\forall x \hspace{5pt} P(x,y)$

при $y=1 \rightarrow \mathds{1}$

при $y=2 \rightarrow \mathds{O}$

~\\
\textbf{Замечание.} Чтобы вычислить значение формулы исчисления предикатов, надо:

\hspace{10pt}-интерпретация, т.е. М-?

Задать смысл предикатов и задать множество функциональных символов.

\hspace{10pt}- задать значения свободных переменных. Без этого мы получим не $\mathds{O}, \mathds{1}$, а предикат.

т.е. $\forall x \hspace{5pt} P(x,y) = Q(y)$ - предикат от у.

~\\
\noindent 5. $\exists x (\forall y \hspace{5pt} P(x,y))$

свободные переменные: нет.

Интерпретации:

1) $M=\mathds{Z} \hspace{5pt} P(x,y): x\leqslant y$

$\forall y \hspace{5pt} P(x,y)=Q(x)$ - обозначим. 

или $Q(x) = \forall y \hspace{5pt} x\leqslant y$.

$Q(0) = \forall y \hspace{5pt} 0\leqslant y = \mathds{O}$

$Q(-1) = \forall y \hspace{5pt} -1\leqslant y = \mathds{O}$

$Q(...) = \forall y \hspace{5pt} ...\leqslant y = \mathds{O}$

$\Rightarrow Q(x) =\mathds{O}$ независимо от х.

$\exists x \forall y P(x,y) = \exists x$  $\mathds{O}=\mathds{O}$. можно подобрать $x\in M$ внутри 1. $\forall y P(x,y) = Q(x)$.

2) другая интерпретация

$M=\mathds{N}$

$P(x,y): x\leqslant y$

$\forall y P(x,y) = Q(x)$.

$Q(1):\forall y \hspace{10pt} 1\leqslant y= \mathds{1}$

$Q(2):\forall y \hspace{10pt} 2\leqslant y= \mathds{O}$

$\exists x Q(x) = \exists x = \left\{ \begin{array}{ll}
         x = 1 & \mbox{$\mathds{1}$};\\
         else  & \mbox{$\mathds{O}$}.\end{array} \right. = \mathds{1}$ - при х=1.
         
~\\
\textbf{Еще примеры.}

$M = \mathds{N}$

$P(x,y): x=y$

$f(x,y)=x+y$

$x>y=\exists k \hspace{10pt} x=y+k = (x,+(y,k))$

$x\vdots y = \exists k \hspace{10pt} x=y*k$









\end{document}
