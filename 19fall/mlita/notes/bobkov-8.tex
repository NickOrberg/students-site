\documentclass{article}

%Russian-specific packages
%--------------------------------------
\usepackage[T2A]{fontenc}
\usepackage[utf8]{inputenc}
\usepackage[russian]{babel}
\usepackage{cancel}
\usepackage{graphicx}
\graphicspath{ {images/} }
\usepackage{amsmath}
%--------------------------------------

%Hyphenation rules
%--------------------------------------
\usepackage{hyphenat}
\hyphenation{ма-те-ма-ти-ка вос-ста-нав-ли-вать}
%--------------------------------------

\title{Лекция по математической логике и теории алгоритмов}
\date{30 октября 2019}

\begin{document}

\maketitle
\begin{center}
\textbf{Исчисление предикатов.}
\end{center}

Исчисление высказываний: есть функция логическая.

Переменные: в функциях - пропорциональные 0 или 1. Легко перебирать все возможные значения переменных.

Исчисление предикатов: переменные - предметные приимают значения = элементы непустого множества, т.е. можно формулировать утверждения про элементы какого-то множества (числа,слова,студенты ...).

Итак,чтобы понять формулировать утверждения,заводим множество.

М $\not =$ 0

\textbf{Определение:} Предикат - это функция 

P: $M^k \longrightarrow \beta$ 

$k \geq 0 , k\in Z$

B = {F,T}, где F - ложь,а T - истина.

Примеры:

M = Z

$P_1 (x) = x \geq 0$, что x положительно

$P_2 (x) = x^2 + 1 \geq 7$

$P_3 (x) = x$ содержит цифру в 10-й записи

$P_3 (238) = F$

$P_3 (-571) = T$

$P_4(x,y) = x > y$

$P_5(x,y) = x^2 + y^2 = 25$

$P_5(7,8) = F$

$P_5(3,4) = T$

$P_5(0,5) = T$

$P_6 = F , P_7 = T$, k=0, нет переменных.

$P_8(x,y) = x$ посещал лекции чаще чем у в этом семестре

М = студенты этого потока.

\textbf{Определение:} Функции - f: $M^k \longrightarrow M$ , $k \geq 0 k \in Z$. Функции превращают один или несколько элементов множества в элемент множества.

Примеры:

М = 7

$f_1(x,y) = x + y$

$f_2(x,y) = x^2 + y - 1$

$f_3(x,y) = 
\begin{cases}
y, &\text{если x - четный}\\
42, &\text{если x - нечетный}
\end{cases}$

$f_3(2,5) = 5$

$f_3(7,8) = 42$

$f_4(x) = x^2$

$f_5(x) = x$ без цифр 1 в 10-й записи.

$f_5(42) = 42$

$f_5(57121) = 572$

$f_5(111) = 0$

$f_6 = 7 \longleftarrow$ константа k=0

$f_7(x,y)$ = третий студент (кроме x,y).

M = {Ниф Ниф, Наф Наф, Нуф Нуф}

Обязательно доопределить

$f_7(x,y) = x$ если x=y.

\textbf{Замечания:}

1)Предикаты - заглавные буквы A,B,C,P,Q,R,S

Функции - строчные буквы f,g,h...

Константы - начало алфавита a,b,c...

2)Некоторые функции и предикаты можно записать привычно в инфиксной форме:

x>y вместо P(x,y) > (x,y)

x + y вместо f(x,y)

$x^y$ вместо g(x,y), где g - возведение в степень.

\textbf{Определение:} 
формула исчисления предикатов содержит: предикатные символы, функциональные символы, предметные переменные,кванторы.

Подопределения.

Термы:переменная функция символ ( , , ).

Пример: Пусть x,y,z - переменные. Пусть f,g,a - функциональные символы.

\textbf{Замечание:}

x f(f(x))

f(x) f(g(x,y))

f(g(x,g(a, f(y))))

Формула исчисления предикатов - это

- предикатный символ (терм, терм, терм) (все переменные внутри термов свободны)

- любое х. Функция исч. предикатов и свободной переменной x. (здесь переменная х перестает быть свободной и становится связанной).

- $\neg$ ФИП, ФиП $\Longrightarrow$ ФИП.

- существует х. -///-

ФИП - это выражение предиката через другие предикаты и функции. При этом смысл функций и предикатов не важен, но если смысл будет задан, то получится конкретный предикат.

Примеры:

1. P(x,y), где x и y - перм (перемен.); P - предикатный символ.

Интерпретация:

Задать M и символ P.

1) M = z

P(x,y) : x > y

2) M = студентов P(x,y) : x чаще ходит.

Обе переменные x,y - свободны. Это значит, что им можно назначить конкретное значение, и тогда результат вычисления истина или ложь.

x = 5, y = 7 : p(x,y) = F

x = 4, y = 2 : P(x,y) = T

2. P(x,a), где x - перем (терм); y - костанта(терм) (функция).

Интерпретация: P = ?, a = ?, m = ?

1) m = z, P(x,y) x > y a = z

одна свободная переменная x

x = 8 : T

x = y : F

3. P(x) $\vee$ Q(x)

Интерпретация:

P = ?

a = ?

1) m = z, P(x) = x - четн.; Q(x) = x - нечет.

свободная переменная: х

х = 1: $F \vee T = T$

x = 10: $T \vee F = T$

x = неважно: ... $\vee$ ... T

4. Для любого х P(x)

Интерпретация: M = ? P = ?

Свободны: y (x - связан)

1) M = z, P(x,y) : x >= y

y = 0.

Для любого P(x,0) $\Longleftrightarrow$ для любого x >= 0

Чтобы вычислить любое x P(x) надо проверить, что P(x) всегда T при всех x $\in$ M

2) M = N P(x,y) : x >= y

Любое х P(x,y)

при $y = 1 \Longrightarrow T$

при $y = 2 \Longrightarrow T$

\textbf{Замечание:}
чтобы вычислить формулы исчисления предметов надо:

- интерпретация, т.е. M = ? задать множество 

P,Q = ?

f, q = ?

задать смысл предикатов и функций символов - задать значения свободных переменных без этого мы получаем не F, T, а предикат. То есть для любого х P(x,y) = Q(y) - предикат от y.

5. Существует х(любое y P(x,y))

свободные переменные: нет

интерпретации:

1) M = z P(x,y) : x <= y

любое y P(x,y) = Q(y) - обозначим

или Q(x) = любое y, x <= y

Q(0) = любое y, 0 <= y = F

Q(-1) = любое y, -1 <= y = F

Q(...) = любое y, ... <= y = F

То есть Q(x) = F независимы от х

Существует х любое y P(x,y) = существует x(можно преобразовать $x \in m$ внутри Т), F = F.

2) другая интерпретация

M = N p(x,y) : x <= y

Существует x любой y P(x,y) = существует x = T при x = 1.

Q(1) = любое y, 1 <= y = F

Q(2) = любое y, 2 <= y = F

Q(...) = любое y, ... <= y = F

Ещё примеры. Начнем с интерпретации. 

M = N

P(x,y) : x = y

F(x,y) : x + y

и ещё несколько стандартных функций и предметов

x > y = существует K : x = y + k = (x + (y*k))

x:y = существует k : x = y + k

x - степень числа 2

x = 1,2,4,8,16,...

\end{document}